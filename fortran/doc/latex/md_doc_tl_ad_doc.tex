\subsection*{Description \+:}

The Tangent Linear and Adjoint model model are implemented in the same way as the nonlinear model, with a tensor storing the different terms. The Tangent Linear (TL) tensor $\mathcal{T}_{i,j,k}^{TD}$ is defined as\+:

\[ \mathcal{T}_{i,j,k}^{TL} = \mathcal{T}_{i,k,j} + \mathcal{T}_{i,j,k} \]

while the Adjoint (AD) tensor $\mathcal{T}_{i,j,k}^{AD}$ is defined as\+:

\[ \mathcal{T}_{i,j,k}^{AD} = \mathcal{T}_{j,k,i} + \mathcal{T}_{j,i,k} . \]

where $ \mathcal{T}_{i,j,k}$ is the tensor of the nonlinear model.

These two tensors are used to compute the trajectories of the models, with the equations

\[ \frac{d\delta y_i}{dt} = \sum_{j=1}^{ndim}\sum_{k=0}^{ndim} \, \mathcal{T}_{i,j,k}^{TL} \, y^{\ast}_k \; \delta y_j . \]

\[ -\frac{d\delta y_i}{dt} = \sum_{j=1}^{ndim} \sum_{k=0}^{ndim} \, \mathcal{T}_{i,j,k}^{AD} \, y^{\ast}_k \; \delta y_j . \]

where $\boldsymbol{y}^{\ast}$ is the point where the Tangent model is defined (with $y_0^{\ast}=1$).

\subsection*{Implementation \+:}

The two tensors are implemented in the module \hyperlink{namespacetl__ad__tensor}{tl\+\_\+ad\+\_\+tensor} and must be initialized inside a given \hyperlink{structmodel__def_1_1model}{model\+\_\+def\+::model} object with the method \hyperlink{namespacemodel__def_a70a2babf6e9897311a4177e864cc5a9e}{model\+\_\+def\+::init\+\_\+tl\+\_\+model} and \hyperlink{namespacemodel__def_a0eb49b66f98539511b388dabbf00586c}{model\+\_\+def\+::init\+\_\+ad\+\_\+model}. The tendencies are then given by the routine \hyperlink{namespacemodel__def_a77f89a429fafffdb6341feda8ff43274}{model\+\_\+def\+::tl\+\_\+tendencies} and \hyperlink{namespacemodel__def_a1622350bace311440f39acb8dc647059}{model\+\_\+def\+::ad\+\_\+tendencies}. Integrators with the Heun method (R\+K2) or the 4th-\/order Runge-\/\+Kutta method are available with the classes \hyperlink{namespacerk2__tl__integrator}{rk2\+\_\+tl\+\_\+integrator}, \hyperlink{namespacerk2__ad__integrator}{rk2\+\_\+ad\+\_\+integrator}, \hyperlink{namespacerk4__tl__integrator}{rk4\+\_\+tl\+\_\+integrator} and \hyperlink{namespacerk4__ad__integrator}{rk4\+\_\+ad\+\_\+integrator}. An example on how to use it can be found in the test file \hyperlink{test__tl__ad_8f90_source}{test\+\_\+tl\+\_\+ad.\+f90} 